\documentclass[conference, 10pt]{IEEEtran}
\usepackage[cmex10]{amsmath}
\usepackage{amssymb}
\usepackage{graphicx}
\usepackage{color}
\usepackage{placeins}
%\usepackage{stfloats}
\usepackage{float}
\usepackage{hyperref}
\usepackage{cite}
\usepackage{tabularx,colortbl}

\begin{document}
\title{\texttt{yacoub}: a Python package for Simulating Generalized Fading Channels}

\author{\IEEEauthorblockN{Jos\'e~V.~de~M.~Cardoso,~Paulo~R.~Lins~Jr,~Wamberto~J.~L.~Queiroz,
                          ~and~Marcelo~S.~Alencar,~\textit{IEEE Senior Member}}
\IEEEauthorblockA{Universidade Federal de Campina Grande\\
Campina Grande, Para\'iba, Brazil\\
\{josevinicius,paulo,wamberto,malencar\}@iecom.org.br\\}
}

\maketitle

\begin{abstract}
    We present a well tested Python-based library for simulating and computing
    generalized fading channels, named \texttt{yacoub}. We describe the
    applicability of \texttt{yacoub} using examples in recent communications
    system, namely: cooperative spectrum sensing, bit error rate computation
    in generalized fading, and free space optics. The development of \texttt{yacoub}
    open source and its code is avaliable at \texttt{http://github.com/mirca/yacoub}.
\end{abstract}

\IEEEpeerreviewmaketitle
\section{Introduction}

\section{The Acceptance-Rejection sampler in log-space}

\section{Examples}
\subsection{Signal detection}
\subsection{Free Space Optics}
\subsection{BER in $\alpha-\mu$ Fading}

\section{Conclusions}


\section*{Acknowledgement}
The authors would like to thank the Federal University of Campina Grande (UFCG)
and the Institute for Advanced Studies in Communications (Iecom).

\bibliographystyle{IEEEtran}
\bibliography{IEEEabrv,aps2016_regular.bib}

\end{document}
