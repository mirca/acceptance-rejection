\documentclass[conference, 10pt]{IEEEtran}
\usepackage[cmex10]{amsmath}
\usepackage{amssymb}
\usepackage{graphicx}
\usepackage{color}
\usepackage{placeins}
\usepackage{bm}
%\usepackage{stfloats}
\usepackage{float}
\usepackage{hyperref}
\usepackage{cite}
\usepackage{tabularx,colortbl}

\begin{document}
\title{\texttt{yacoub}: a Python package for Simulating Generalized Fading Channels}

\author{\IEEEauthorblockN{Jos\'e~V.~de~M.~Cardoso,~Paulo~R.~Lins~J\'unior,
                          Wamberto~J.~L.~Queiroz,}
        \IEEEauthorblockN{~Francisco~M.~Bernardino~J\'unior,~and~Marcelo~S.~Alencar,~\textit{IEEE Senior Member}}
\IEEEauthorblockA{Universidade Federal de Campina Grande\\
Instituto Federal de Educa\c c\~ao, Ciencia, e Tecnologia da Para\'iba\\
Campina Grande, Para\'iba, Brasil\\
Universidade de Pernambuco\\
Recife, Brasil\\
\{josevinicius,paulo,malencar\}@iecom.org.br\\}
}

\maketitle

\begin{abstract}
    We present a well tested Python-based library for simulating and computing
    generalized fading channels, named \texttt{yacoub}. We describe the
    applicability of \texttt{yacoub} using examples in recent communications
    systems challenges, namely: cooperative spectrum sensing, bit error rate computation
    in generalized fading channel, and parameter estimation in free space optics.
    The development of \texttt{yacoub} open source and its code is avaliable at
    \texttt{http://github.com/mirca/yacoub}.
\end{abstract}

\IEEEpeerreviewmaketitle
\section{Introduction}

\subsection{Note on notation}
Scalars and random variables are denoted as \textit{italic} small-case letters \textit{e.g.} $x$;
vectors and random vectors are denoted as \textit{italic}, boldface, small-case letters \textit{e.g.} $\bm{x}$.
A complex vector of length $n$ is defined as $\bm{x} \in \mathbb{C}^{1\times n}$. All
vectors are column vectors. Matrices are denoted as \textit{italic}, boldface, capital
letters as in $\bm{X}$; the identity matrix of order $n$ is denoted as $\bm{I}_n$.
We define a discrete-time circularly symmetric Gaussian process $\bm{z}$ as any (finite or infinite) collection
of random varibles $\bm{z}~=~\bm{x}~+~j\bm{y}$, $j \triangleq \sqrt{-1}$, such that
$\bm{x}$ and $\bm{y}$ are iid jointly Gaussian with zero mean vector and covariance
matrix given by $\mathbb{E}\left[\bm{z}\bm{z}^{\dagger}\right]$, in which $\bm{z}^\dagger$
means the conjugate transpose of $\bm{z}$.

\section{The Acceptance-Rejection sampler in log-space}

\section{Examples}
\subsection{Spectrum Sensing in Complex Generalized Fading Channels}

The spectrum sensing problem consists in deciding whether or not a given channel
frequency band is being occupied by a licensed (primary) user and, in case that such
frequency band is available, how to opportuniscally allocate secondary users
such that...

From a probabilistic point of view, the spectrum sensing problem may be framed as
a decision theory problem, as follows
\begin{align}
    \mathcal{H}_0:~& \bm{y} = \bm{w},\\
    \mathcal{H}_1:~& \bm{y} = h\bm{s} + \bm{w},
\end{align}
in which $\bm{y} \in \mathbb{C}^{1\times n}$ is the decoded received vector,
 $\bm{w} \in \mathbb{C}^{1\times n}$ is complex Gaussian noise process with
 zero mean vector and covariance matrix given as $\sigma^2\bm{I}_n$

\subsection{Parameter Estimation in Free Space Optics}

\subsection{BER in Complex $\alpha-\mu$ Fading}

\section{Conclusions}

\section*{Acknowledgement}
The authors would like to thank the Federal University of Campina Grande (UFCG)
and the Institute for Advanced Studies in Communications (Iecom).

\bibliographystyle{IEEEtran}
\bibliography{IEEEabrv, yacoub.bib}

\end{document}
