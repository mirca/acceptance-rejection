\documentclass[conference, 10pt]{IEEEtran}
\usepackage[cmex10]{amsmath}
\usepackage{amssymb}
\usepackage{graphicx}
\usepackage{color}
\usepackage{placeins}
\usepackage{bm}
\usepackage{cite}
%\usepackage{stfloats}
\usepackage{float}
\usepackage{hyperref}
\usepackage{cite}
\usepackage{tabularx,colortbl}

\begin{document}
\title{\texttt{maoud}: a Python package for Simulating Generalized Fading Channels}

\author{\IEEEauthorblockN{Jos\'e~V.~de~M.~Cardoso,~Wamberto~J.~L.~Queiroz,}
        \IEEEauthorblockN{~Paulo~R.~Lins~J\'unior,~and~Marcelo~S.~Alencar,~\textit{IEEE Senior Member}}
\IEEEauthorblockA{Universidade Federal de Campina Grande\\
Instituto Federal de Educa\c c\~ao, Ciencia, e Tecnologia da Para\'iba\\
Campina Grande, Para\'iba, Brasil\\
\{josevinicius,paulo,wamberto,malencar\}@iecom.org.br\\}
}

\maketitle

\begin{abstract}
    We present a well tested Python-based library for simulating and computing
    generalized fading channels, named \texttt{maoud}. We describe the
    applicability of \texttt{maoud} using examples in scenarios of communications
    channels impaired by generalized fading, namely: spectrum sensing, bit error
    rate computation, and fading estimation. For the latter, we develop an iterative
    algorithm using the Majorization-Minimization framework, which allows reliable
    estimation of the fading parameter. The development of \texttt{maoud} is open
    source and its code along with examples are avaliable at
    \texttt{http://github.com/mirca/maoud}.
\end{abstract}

\IEEEpeerreviewmaketitle
\section{Introduction}

\subsection*{Notation}
Scalars and random variables are denoted as \textit{italic}, small-case letters
\textit{e.g.} $x$; sets and events are denoted as \textit{italic}, capital
letters \textit{e.g.}, $A$; vectors and random vectors are denoted as
\textit{italic}, boldface, small-case letters \textit{e.g.} $\bm{x}$. The $n-$th
component of a vector $\bm{x}$ is denoted as $x_n$. A complex vector of length
$n$ is defined as $\bm{x} \in \mathbb{C}^{n\times 1}$. All vectors are column
vectors. Matrices are denoted as \textit{italic}, boldface, capital letters as
in $\bm{X}$; the identity matrix of order $n$ is denoted as $\bm{I}_n$. We
define a discrete-time circularly symmetric Gaussian process $\bm{z}$ as any
collection of random variables $\bm{z}~=~\bm{x}~+~j\bm{y}$,
$j \triangleq \sqrt{-1}$, such that $\bm{x}$ and $\bm{y}$ are i.i.d. jointly
Gaussian, with zero mean vector and covariance matrix given by
$\mathbb{E}\left[\bm{z}\bm{z}^{\dagger}\right]$, in which $\bm{z}^\dagger$ means
the conjugate transpose of $\bm{z}$. The expectation value w.r.t. to the
probability distribution of a random variable $x$ is denoted as $\mathbb{E}_x$.
The probability of an event $A$ is denoted as $\mathbb{P}(A)$. The indicator
function is denoted as $\mathbb{I}(\cdot)$, it evaluates to one if its argument
is true and zero otherwise. For any given two real functions $f$ and $g$ defined
on the same domain $D$, $f \cong g$ means that there exist a constant $c$ such that
$f(\bm{x}) = g(\bm{x}) + c$, $\forall~ \bm{x} \in D$.

\section{The Acceptance-Rejection sampler in log-space}

\section{Majorization-Minimization Algorithms}

\section{Examples}
\subsection{Spectrum Sensing in Complex Generalized Fading Channels}

The spectrum sensing problem consists in deciding whether or not a given channel
frequency band is being occupied by a licensed (primary) user and, in case that such
frequency band is available, how to opportuniscally allocate secondary users
such that the interference on the primary user is negligible.

From a probabilistic point of view, the spectrum sensing problem may be framed as
a decision theory problem, as follows
\begin{align}
    H_0:~& \bm{y} = \bm{w},\\
    H_1:~& \bm{y} = h\bm{s} + \bm{w},
\end{align}
in which $\bm{y} \in \mathbb{C}^{n\times 1}$ is the decoded received vector signal,
$\bm{w} \in \mathbb{C}^{n\times 1}$ is complex Gaussian noise process with zero mean
vector and covariance matrix given as $\sigma^2\bm{I}_n$, and $h$ is the channel gain.

In~\cite{cardoso2017}, the authors have shown that the probability distribution of the
energy statistic $\tilde{y} \triangleq \bm{y}^{\dagger}\bm{y}$ conditioned on the knowledge of $h$,
in case that $\bm{s}$ is an $M$-PSK signal such that every symbol has the same probability of occurrence,
$\mathbb{P}(s_n = s) = \frac{1}{M}$, is given as
\begin{align}
    p(\tilde{y} | h, H_1) = 1 - Q_{n}\left(\sqrt{\dfrac{2n|h|^2E_s}{\sigma^2}}, \sqrt{\dfrac{2\tilde{y}}{\sigma^2}}\right),
\end{align}
in which $Q_{n}$ is the Marcum-$Q$ function and $E_s$ is the energy per symbol.

The pdf of $\tilde{y}$ can be written using the Law of Total Expectation
\begin{align}
    p(\tilde{y} | H_1) = \mathbb{E}_{h}\left[p(\tilde{y} | h, H_{1})\right]
                 = \int_{-\infty}^{+\infty} p(\tilde{y} | h, H_1)p(h)\;\mathrm{d}h.
\end{align}

Recall that the energy detection rule can be expressed as
\begin{equation}
    d_\delta (\tilde{y}) = \mathbb{I}(\tilde{y} > \delta)
\right.
\end{equation}
in which $\delta$ is a strictly positive real number known as energy threshold,
and $d_\delta (\tilde{y}) = j,~j \in \{0,1\}$, means that the detector has decided
in favor of the hypothesis $H_j$.

As a result, the probabilities of false alarm and miss detection can
be written as
\begin{align}
    p_f &\triangleq \mathbb{P}\left(d_\delta(\tilde{y}) = 1 | H_0\right) = 1 -  p(\delta | H_0),\label{eq:pf} \\
    p_d &\triangleq \mathbb{P}\left(d_\delta(\tilde{y}) = 0 | H_1\right) = \mathbb{E}_{h}\left[p(\delta, h | H_1)\right],
\label{eq:pd}
\end{align}

\subsection{Parameter Estimation in Nakagami-$m$ fading}
The Nakagami-$m$ density is given as
\begin{align}
    p(\bm{h})& = \prod_{i=1}^{n}\dfrac{2m^m}{\Gamma(m)\Omega^{m}}h_i^{2m - 1}
              \exp\left(-\dfrac{mh_i^2}{\Omega}\right) \nonumber \\
          = & \left(\dfrac{2m^m}{\Gamma(m)\Omega^{m}}\right)^{n}
          \exp\left(-\dfrac{m\sum_{i=1}^{n}h_i^2}{\Omega}\right) \prod_{i=1}^{n}h_i^{2m - 1}
\end{align}
And the log likelihood function (up to an additive constant) is given as
\begin{align}
\log p(\bm{h}) \cong &~n\left(m\left(\log m - \log\Omega\right) - \log\Gamma(m)\right)\nonumber
    \\ & -m\left(\dfrac{\sum_{i=1}^{n}h_i^2}{\Omega} - 2\sum_{i=1}^{n}\log h_i\right)
    \label{eq:loglike}
\end{align}

A direct maximum likelihood estimator (MLE) for (\ref{eq:loglike}) has been
investigated to be infeasible~\cite{cheng2001}.

Therefore, we use a Majorization-Minimization algorithm to find smooth
and easy to optimize lower bounds for $\log p(\bm{h})$. More precisely,
we need to find lower bounds for $m\log m$ and $-\log \Gamma(m)$ for $m \geq \frac{1}{2}$.
The former function is convex for $m \geq 0$, hence it can be lower bounded
by its first order Taylor series as
\begin{align}
    m \log m \geq m(1 + \log m_t) - m_t,
    \label{eq:lower-bound-mlogm}
\end{align}
The function $-\log \Gamma(m)$ is concave, therefore, it can be lower bounded
by its second order Taylor series expansion as
\begin{align}
    -\log \Gamma(m) \geq& - \log \Gamma(m_t) - \psi(m_t) (m - m_t)\nonumber\\
                        & - \frac{\psi'\left(\frac{1}{2}\right)}{2}(m - m_t) ^ 2,
                        ~m \geq \frac{1}{2},~m_t \geq \frac{1}{2},
    \label{eq:lower-bound-negloggamma}
\end{align}
in which $\psi(x) = \dfrac{\Gamma'(x)}{\Gamma(x)}$ is known as the digamma function.
For both inequalities presented above, equality is achieved at $m = m_t$.

Substituting (\ref{eq:lower-bound-mlogm}) and (\ref{eq:lower-bound-negloggamma}) into
(\ref{eq:loglike}), a lower bound for $\log p(\bm{h})$ is obtained in~(\ref{eq:surrogate}).
Due to the simple form of $g(m | m_t)$, its maximizer can be found in closed form, and
an updating rule for the MLE can be written as
\begin{align}
    m_{t+1} = 1 + \dfrac{1}{\psi'(\frac{1}{2})}&\left(1 + \log \frac{m_t}{\Omega} - \psi(m_t)
    + \right.\nonumber\\
    &+ \left.\dfrac{1}{n} \left(2\sum_{i=1}^{n}\log h_i - \frac{\sum_{i=1}^{n}h_i^2}{\Omega}\right)\right)
\end{align}

Additionally,
note that $m_{t+1}$ is an estimator of the true parameter value $m$. Its expected value,
conditioned on the knowledge of $m_t$, is given as
\begin{align}
 \mathbb{E}(m_{t+1} | m_t) = 1 + \dfrac{1}{\psi'(\frac{1}{2})}\left(\log \frac{m_t}{\Omega} - \psi(m_t)
    + \dfrac{2}{n}\sum_{i=1}^{n}\mathbb{E}(\log h_i)\right)
\end{align}

Its variance, also conditioned on the knowledge of $m_t$, is given as
\begin{align}
    \mathrm{var}(m_{t+1} | m_t) = \left(\dfrac{1}{n\psi'(\frac{1}{2})}\right)^2
    \mathrm{var}\left(2\sum_{i=1}^{n}\log h_i
    - \frac{\sum_{i=1}^{n}h_i^2}{\Omega}\right).
\end{align}

The Cram\'er-Rao Lower Bound for any unbiased estimator of $m$, say $\hat{m}$, is given as~\cite{cheng2001}
\begin{align}
    \mathrm{var}(\hat{m}) = \dfrac{1}{n\left(\psi'(m) - \frac{1}{m}\right)}
\end{align}

\begin{figure*}[!htb]
\begin{align}
    g(m | m_t) = n\left(-m\log\Omega  + m(1 + \log m_t) - m_t
    -\log \Gamma(m_t) - \psi(m_t) (m - m_t) - \frac{\psi'\left(\frac{1}{2}\right)}{2}(m - m_t)^2\right)
    -m\left(\dfrac{\sum_{i=1}^{n}h_i^2}{\Omega} - 2\sum_{i=1}^{n}\log h_i\right)
    \label{eq:surrogate}
\end{align}
\end{figure*}

\subsection{BER in Complex $\alpha-\mu$ Fading}

Consider the system
\begin{align}
    \bm{y} = h\bm{s} + \bm{w}
\end{align}
in which $\bm{s} \in \mathbb{C}^{n\times 1}$ is a complex On-Off Keying (OOK) signal,
$h$ is a complex $\alpha-\mu$ random variable and $\bm{w}$ is a complex Gaussian process
with zero mean vector and covariance matrix equals $\sigma^2\bm{I}_n$, and $\bm{y}$ is
the received complex vector signal.

Assume that the OOK symbols are equiprobable and that there exist no interference
between the in-phase and quadrature components, then the probability of one bit error
is given as
\begin{align}
    p_{e} = \dfrac{1}{2}\left(\mathbb{P}\left(\hat{y}_i = 0 | s_{i} = 1\right)
                            + \mathbb{P}\left(\hat{y}_i = 1 | s_{i} = 0\right)\right).
\end{align}

Assume that the decoded vector $\bm{\hat{y}}$ is estimated using the minimum distance decoding
rule, i.e.,

\section{Conclusions}

\section*{Acknowledgement}
The authors would like to thank the Federal University of Campina Grande (UFCG)
and the Institute for Advanced Studies in Communications (Iecom) for supporting
this research.

\bibliographystyle{IEEEtran}
\bibliography{manuscript.bib}

\end{document}
